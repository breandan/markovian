%%%%%%%%%%%%%%%%%%%%%%%%%%%%%%%%%%%%%%%%%%%
%
% From a template maintained at https://github.com/jamesrobertlloyd/cbl-tikz-poster
%
% Code near the top should be fairly standard and not need to be changed
%  - except for the document class
% Code lower down is more likely to be customised
%
%%%%%%%%%%%%%%%%%%%%%%%%%%%%%%%%%%%%%%%%%%%

%%%%%%%%%%%%%%%%%%%%%%%%%%%%%%%%%%%%%%%%%%%
%
% Document class
%
% Change this if you want a different size / orientation poster etc
%
%%%%%%%%%%%%%%%%%%%%%%%%%%%%%%%%%%%%%%%%%%%

\documentclass[landscape,a0b,final,a4resizeable]{a0poster}
%\documentclass[portrait,a0b,final,a4resizeable]{a0poster}

%%%%%%%%%%%%%%%%%%%%%%%%%%%%%%%%%%%%%%%%%%%
%
% 'Basic' packages
%
% TODO - Almost certainly some are unnecessary - feel free to remove nonstandard
% packages if you think it is a good idea not to always have them
%
%%%%%%%%%%%%%%%%%%%%%%%%%%%%%%%%%%%%%%%%%%%

\usepackage{qrcode}
\usepackage{multicol}
\usepackage{color}
\usepackage{shadow}
\usepackage{morefloats}
\usepackage{cite}
\usepackage[pdftex]{graphicx}
\usepackage{rotating}
\usepackage{amsmath, amsthm, amssymb, bm}
\usepackage{array}
\usepackage{nth}
\usepackage[square,numbers]{natbib}
\usepackage{booktabs}

%%%%%%%%%%%%%%%%%%%%%%%%%%%%%%%%%%%%%%%%%%%
%
% TIKZ packages and common definitions
%
% Add extra things as per your tikz needs
%
%%%%%%%%%%%%%%%%%%%%%%%%%%%%%%%%%%%%%%%%%%%

\usepackage{a0size}
\usepackage{tikz}
\usetikzlibrary{shapes.geometric,arrows,chains,matrix,positioning,scopes,calc}
\tikzstyle{mybox} = [draw=white, rectangle]

%%%%%%%%%%%%%%%%%%%%%%%%%%%%%%%%%%%%%%%%%%%
%
% myfig
%
% \myfig - replacement for \figure
% necessary, since in multicol-environment
% \figure won't work
%
%%%%%%%%%%%%%%%%%%%%%%%%%%%%%%%%%%%%%%%%%%%

\newcommand{\myfig}[3][0]{
\begin{center}
    \vspace{1.5cm}
    \includegraphics[width=#3\hsize,angle=#1]{#2}
    \nobreak\medskip
\end{center}}

%%%%%%%%%%%%%%%%%%%%%%%%%%%%%%%%%%%%%%%%%%%
%
% mycaption
%
% \mycaption - replacement for \caption
% necessary, since in multicol-environment \figure and
% therefore \caption won't work
%
%%%%%%%%%%%%%%%%%%%%%%%%%%%%%%%%%%%%%%%%%%%

%\newcounter{figure}
\setcounter{figure}{1}
\newcommand{\mycaption}[1]{
\vspace{0.5cm}
\begin{quote}
{{\sc Figure} \arabic{figure}: #1}
\end{quote}
\vspace{1cm}
\stepcounter{figure}
}

%%%%%%%%%%%%%%%%%%%%%%%%%%%%%%%%%%%%%%%%%%%
%
% Some standard colours
%
%%%%%%%%%%%%%%%%%%%%%%%%%%%%%%%%%%%%%%%%%%%

\definecolor{camlightblue}{rgb}{0.601 , 0.8, 1}
\definecolor{camdarkblue}{rgb}{0, 0.203, 0.402}
\definecolor{camred}{rgb}{1, 0.203, 0}
\definecolor{camyellow}{rgb}{1, 0.8, 0}
\definecolor{lightblue}{rgb}{0, 0, 0.80}
\definecolor{white}{rgb}{1, 1, 1}
\definecolor{whiteblue}{rgb}{0.80, 0.80, 1}

%%%%%%%%%%%%%%%%%%%%%%%%%%%%%%%%%%%%%%%%%%%
%
% Some look and feel definitions
%
%%%%%%%%%%%%%%%%%%%%%%%%%%%%%%%%%%%%%%%%%%%

\setlength{\columnsep}{0.03\textwidth}
\setlength{\columnseprule}{0.0018\textwidth}
\setlength{\parindent}{0.0cm}

%%%%%%%%%%%%%%%%%%%%%%%%%%%%%%%%%%%%%%%%%%%
%
% \mysection - replacement for \section*
%
% Puts a pretty box around some text
% TODO - any other thoughts for what this box should look like
%
%%%%%%%%%%%%%%%%%%%%%%%%%%%%%%%%%%%%%%%%%%%

\tikzstyle{mysection} = [rectangle,
draw=none,
shade,
outer color=gray!30,
inner color=gray!30,
text width=0.965\columnwidth,
text centered,
rounded corners=20pt,
minimum height=0.11\columnwidth]

\newcommand{\mysection}[1]
{
\begin{center}
    \begin{tikzpicture}
        \node[mysection] {\sffamily\bfseries\LARGE#1};
    \end{tikzpicture}
\end{center}
}

%%%%%%%%%%%%%%%%%%%%%%%%%%%%%%%%%%%%%%%%%%%
%
% Set the font
%
% TODO - Not sure what a canonical choice is - feel free to modify
%
%%%%%%%%%%%%%%%%%%%%%%%%%%%%%%%%%%%%%%%%%%%

\renewcommand{\familydefault}{cmss}
\sffamily

%%%%%%%%%%%%%%%%%%%%%%%%%%%%%%%%%%%%%%%%%%%
%
% Poster environment
%
% Centres everything and can be used to define the width of the content
%
%%%%%%%%%%%%%%%%%%%%%%%%%%%%%%%%%%%%%%%%%%%

\newenvironment{poster}{
\begin{center}
\begin{minipage}[c]{0.96\textwidth}
}{
\end{minipage}
\end{center}
}

%%%%%%%%%%%%%%%%%%%%%%%%%%%%%%%%%%%%%%%%%%%
%
% This is probably a good place to put content specific packages and definitions
%
%%%%%%%%%%%%%%%%%%%%%%%%%%%%%%%%%%%%%%%%%%%

%\usepackage{microtype}
%\usepackage{graphicx}
%\usepackage{subfigure}
%\usepackage{booktabs} % for professional tables
%\usepackage{enumitem}
%
\usepackage{bussproofs}
%
\usepackage{algpseudocode}
\usepackage{algorithmicx}
%\usepackage{sourcecodepro}
%\usepackage{listings}
%\usepackage{amsfonts}
%\usepackage{amssymb}
%\usepackage{tikz-qtree}
%\usepackage{amsthm}
%\usepackage{bm}
%\usetikzlibrary{bayesnet}
%\usetikzlibrary{arrows}
%\usepackage{caption}
%\usepackage{subcaption}
%\usetikzlibrary{backgrounds}
%
%\newcommand{\E}{\mathbb{E}}
%\newcommand{\Var}{\mathrm{Var}}
%\newcommand{\Cov}{\mathrm{Cov}}
%
%\usepackage{mathtools}% superior to amsmath
%\usepackage{tikz}
%\tikzset{latent/.append style={minimum size=14pt, inner sep=1pt, node distance=10pt}, every node/.append style={draw,circle, inner sep=1pt}}
%\makeatletter
%\newcommand\ccirc[1]{%
%\mathpalette\@ccirc{#1}%
%}
%\newcommand\@ccirc[2]{%
%\tikz[baseline=(math.base)] \node (math) {$\m@th#1#2$};%
%}
%\newcommand\gcirc[1]{%
%\mathpalette\@gcirc{#1}%
%}
%\newcommand\@gcirc[2]{%
%\tikz[baseline=(math.base)] \node[fill=gray!30] (math) {$\m@th#1#2$};%
%}
%\makeatother

\newtheorem{thm}{Theorem}%[section]
\newtheorem{lem}[thm]{Lemma}
\newtheorem{prop}[thm]{Proposition}
\newtheorem{cor}[thm]{Corollary}

\newtheorem*{theorem*}{Theorem}

\theoremstyle{definition}
\newtheorem*{definition*}{Definition}
\newtheorem{definition}[thm]{Definition}%[section]
\newtheorem{conj}{Conjecture}[section]
\newtheorem{exmp}{Example}[section]
\newtheorem{rem}[thm]{Remark}

\theoremstyle{remark}
%\newtheorem{rem}{Remark}
\newtheorem{note}{Note}
\newtheorem{case}{Case}

\newcommand{\eqd}{\overset{\,_{\!d}}{=}}
\newcommand{\defn}[1]{\emph{#1}}

\newcommand{\Law}{\mathcal{L}}

\def\given{\,|\,}

\def\SGinf{\mathbb{S}_{\infty}}

\newcommand{\NonNegInts}{\mathbb{Z}_+}
\newcommand{\Nats}{\mathbb{N}}
\newcommand{\Rationals}{\mathbb{Q}}
\newcommand{\Reals}{\mathbb{R}}

\newcommand{\as}{\textrm{a.s.}}

\def\[#1\]{\begin{align}#1\end{align}}
\newcommand{\defas}{:=}

\newcommand{\Normal}{\mathcal{N}}
\newcommand{\dist}{\ \sim\ }

\newcommand{\kernel}{\kappa}
\newcommand{\kernelmatrix}{K}
\newcommand{\scalefactor}{s}
\newcommand{\lengthscale}{\ell}
\newcommand{\targets}{T}
\newcommand{\noise}{\sigma_\targets}
\newcommand{\pseudopoints}{\eta}
\newcommand{\inputpoints}{\xi}
\newcommand{\covhyppar}{\psi}
\newcommand{\logistic}{\phi}

\newcommand{\CompOrder}{\mathcal{O}}
\def\graphspace{\mathbf{G}}
\def\Uniform{\mbox{\rm Uniform}}
\def\Gaussian{\mbox{\rm Gaussian}}
\def\Bernoulli{\mbox{\rm Bernoulli}}
\def\Dirichlet{\mbox{\rm Dirichlet}}
\def\ie{i.e.,\ }
\def\eg{e.g.,\ }
\def\iid{i.i.d.\ }
\def\simiid{\sim_{\mbox{\tiny iid}}}
\def\simind{\sim_{\mbox{\tiny ind}}}
\def\eqdist{\stackrel{\mbox{\tiny d}}{=}}
\def\ahfunction{\theta}
\def\AHfunction{\Theta}           % A-H random function
\def\AHvar{U}                     % A-H uniform variables
\def\AHvaralt{V}                  % A-H uniform variables - for bipartite data
\def\larray{W}                    % latent array sampled with A-H
%\def\latentspace{\mathbf{W}}      % range of entries
\def\latentspace{\mathcal{W}}      % range of entries
\def\darray{X}                    % data array
%\def\dataspace{\mathbf{X}}        % sample space
\def\dataspace{\mathcal{X}}        % sample space
\def\cfspace{\mathbf{C}}          % space of continuous functions
%\def\GP{\mbox{\mathcal{GP}}}
\def\GP{\mathcal{GP}}
\def\likelihood{P}
\def\CovData{C}
\def\CovDataAlt{D}

\def\newarrow{\mbox{\begin{tikzpicture}
\useasboundingbox{(-3pt,-4.5pt) rectangle (19pt,1pt)};
\draw[->] (0,-0.07)--(17pt,-0.07);\end{tikzpicture}}}

%%%%%%%%%%%%%%%%%%%%%%%%%%%%%%%%%%%%%%%%%%%
%
% The document environment starts here
%
%%%%%%%%%%%%%%%%%%%%%%%%%%%%%%%%%%%%%%%%%%%

\begin{document}

%%%%%%%%%%%%%%%%%%%%%%%%%%%%%%%%%%%%%%%%%%%
%
% Begin the poster environment - centres things and potentially changes the width
%
%%%%%%%%%%%%%%%%%%%%%%%%%%%%%%%%%%%%%%%%%%%

\begin{poster}

%%%%%%%%%%%%%%%%%%%%%%%%%%%%%%%%%%%%%%%%%%%
%
% Potentially add some space at the top of the poster
%
%%%%%%%%%%%%%%%%%%%%%%%%%%%%%%%%%%%%%%%%%%%

\vspace{0\baselineskip}

%%%%%%%%%%%%%%%%%%%%%%%%%%%%%%%%%%%%%%%%%%%
%
% Draw the header as a TIKZ picture
%
% Using TIKZ to allow for easy alignment
%
%%%%%%%%%%%%%%%%%%%%%%%%%%%%%%%%%%%%%%%%%%%

\begin{center}
\begin{tikzpicture}[x=0.5\textwidth]
% Dummy nodes at edges for spacing
% TODO - a better way?
%\node at (+1, 0) {};
%\node at (-1, 0) {};
% Set the size of the badges
\def \badgeheight {0.06\textwidth}
% Title text
\node[inner sep=0,text width=0.5\textwidth, font=\Huge] (Title) at (0,0)
{
{\sffamily \Huge \textbf{Probabilistic Reasoning, from Graphs to Circuits}}\\
{\huge\sffamily Breandan Considine}\\
\vspace{-0.3\baselineskip}
%{\large\sffamily 1: McGill University, School of Computer Science, 2: Knowledge and Software Technology Lab, 3: Institut Qu\'eb\'ecois d'Intelligence Artificielle}
};

% Cambridge badge
\node [mybox] (Cambridge Badge) at (-0.85, 0) {
\includegraphics[height=\badgeheight]{mcgill.png}
};
% CBL badge
%\node [mybox] (CBL Badge) at (-0.7, 0) {
%\includegraphics[height=\badgeheight]{mila.png}
%};
% Columbia logo
\node [mybox] (box) at (0.6, 0) {
\includegraphics[height=\badgeheight]{mila.png}
};
\end{tikzpicture}
\end{center}

%%%%%%%%%%%%%%%%%%%%%%%%%%%%%%%%%%%%%%%%%%%
%
% Spacing between title and main body
%
%%%%%%%%%%%%%%%%%%%%%%%%%%%%%%%%%%%%%%%%%%%

\vspace{1\baselineskip}

%%%%%%%%%%%%%%%%%%%%%%%%%%%%%%%%%%%%%%%%%%%
%
% Columns environment
%
%%%%%%%%%%%%%%%%%%%%%%%%%%%%%%%%%%%%%%%%%%%

\begin{multicols}{3}

%%%%%%%%%%%%%%%%%%%%%%%%%%%%%%%%%%%%%%%%%%%
%
% Start of content
%
%%%%%%%%%%%%%%%%%%%%%%%%%%%%%%%%%%%%%%%%%%%

\large

\mysection{Abstract}

\begin{itemize}
\item Modeling requires lots of pen-and-paper derivation to derive efficient estimators
\item Once the estimator has been derived, programmer implements the entire model
\item Developers must reimplement many ad-hoc algorithms for each application
\item Is it possible to automate probabilistic inference like AD did for deep learning?
\item Prior work in probabilistic programming focuses heavily on numerical methods
\end{itemize}

\\

%\begin{center}
%\input{../misc/interactome_adj.tex}
%\end{center}

\mysection{Denotational Semantics}

\vspace{\baselineskip}

The grammar of probabilistic modeling can be described approximately as follows:

\vspace{\baselineskip}

\begin{center}
\begin{tabular*}{0.3\textwidth}{l @{\extracolsep{\fill}} l @{\extracolsep{\fill}} l @{\extracolsep{\fill}} l}
$\mathcal{D} \rightarrow \Gaussian$  & $\mathcal{V} \rightarrow \mathcal{V}, \mathcal{V}$      & $\mathcal{S} \rightarrow \mathcal{V} \sim \mathcal{D}$ & \mathcal{P} \rightarrow t \\
$\mathcal{D} \rightarrow \Bernoulli$ & $\mathcal{P} \rightarrow P(\mathcal{V})$                & $\mathcal{E} \rightarrow \mathcal{V} + \mathcal{V}$    & \mathcal{P} \rightarrow t \\
$\mathcal{D} \rightarrow \Dirichlet$ & $\mathcal{P} \rightarrow P(\mathcal{V}\mid\mathcal{V})$ & $\mathcal{E} \rightarrow \mathcal{V}\times\mathcal{V}$ & \mathcal{P} \rightarrow t \\
$\mathcal{V} \rightarrow A \mid \ldots \mid Z$ &  $\mathcal{P} \rightarrow P(\mathcal{E})$ & & \\
\end{tabular*}
\end{center}

\vspace{\baselineskip}

Given a distribution over a set $X$, we can \textit{sample} from it to produce a single element from that set, a \textit{random variable}:

\vspace{\baselineskip}

\begin{prooftree}
\AxiomC{$\Gamma \vdash P(X): X \rightarrow \mathbb{R}^{+}$}
\AxiomC{$x \sim P(X)$}
\RightLabel{Sample}
\BinaryInfC{$\Gamma \vdash x: (X\rightarrow \mathbb{R}^{+}) \leadsto X$}
\end{prooftree}

\vspace{\baselineskip}

%We can assign a probability distribution to a variable, by \textit{sampling} from it. As we increase the sample size, the distribution of values will converge to the true distribution.
%
%$$
%d \sim \mathcal{D}
%$$

The joint distribution $P(X, Y)$ is a distribution over the Cartesian product of the sets $X$ and $Y$, denoted $X \times Y$:

\vspace{\baselineskip}

\begin{prooftree}
\AxiomC{$\Gamma \vdash P(X): X \rightarrow \mathbb R^{+}$}
\AxiomC{$\Gamma \vdash P(Y): Y \rightarrow \mathbb R^{+}$}
\RightLabel{Joint}
\BinaryInfC{$\Gamma \vdash P(X, Y): X \times Y \rightarrow \mathbb R^{+}$}
\end{prooftree}

\vspace{\baselineskip}

Given a joint distribution $P(X, Y)$, if we see an event $y: Y$, this observation is called \textit{conditioning} and the resulting distribution over $X$ is called a \textit{conditional distribution}:

\vspace{\baselineskip}
% https://www.cambridge.org/core/services/aop-cambridge-core/content/view/819623B1B5B33836476618AC0621F0EE/9781108488518AR.pdf/Foundations_of_Probabilistic_Programming.pdf?event-type=FTLA#page=325

\begin{prooftree}
\AxiomC{$\Gamma \vdash P(X, Y): X \times Y\rightarrow\mathbb{R}^+$}
\AxiomC{$\Gamma \vdash y: Y$}
\RightLabel{Cond}
\BinaryInfC{$\Gamma \vdash P(X \mid Y = y): X\rightarrow\mathbb{R}^+$}
\end{prooftree}

\vspace{\baselineskip}

We can use Bayes' rule to exchange the order of conditioning as follows:

\vspace{\baselineskip}

\begin{prooftree}
\AxiomC{$P(X \mid Y)$}
\AxiomC{$P(Y)$}
\RightLabel{Bayes}
\BinaryInfC{$P(Y \mid X) \propto P(X \mid Y)P(Y)$}
\end{prooftree}

\vspace{\baselineskip}

When a conditional distribution $P(X \mid Y)$ does not depend on its prior $Y$, or equivalently may be factorized, we may conclude that $X$ and $Y$ are independent:

\vspace{\baselineskip}

\begin{prooftree}
\AxiomC{$P(X \mid Y) = P(X)$}
\RightLabel{Ind\hspace{100pt}}
\UnaryInfC{$X \perp Y$}
\DisplayProof
\AxiomC{$P(X, Y) = P(X)P(Y)$}
\RightLabel{Fact}
\UnaryInfC{$X \perp Y$}
\end{prooftree}

\vspace{\baselineskip}

When two conditionals $P(X \mid Z)$, $P(Y \mid Z)$ are multiplied to form a joint distribution $P(X, Y \mid Z)$, $X$ and $Y$ are said to be \textit{conditionally independent given $Z$}:

\vspace{\baselineskip}

\begin{prooftree}
\AxiomC{$P(X,Y \mid Z) = P(X \mid Z)P(Y \mid Z)$}
\RightLabel{CondInd}
\UnaryInfC{$X \perp Y \mid Z$ }
\end{prooftree}

\newpage

\mysection{Probabilistic Circuits}

\vspace{1\baselineskip}

%By defining some elementary distributions and composing them by means of simple operators, we obtain a rich framework for probabilistic modeling. Recent work~\citep{choi2020probabilistic} has explored tractable models for probabilistic reasoning based on semiring algebras. Semirings are known to have many useful applications in graph theory~\citep{dolan2013fun} and formal languages~\citep{bernady2013efficient}.

A semiring algebra has two operators, $\oplus$ and $\otimes$, with the usual properties. In particular, distributivity holds:

\begin{prooftree}
\AxiomC{$X \otimes (Y \oplus Z)$}
\UnaryInfC{$(X \otimes Y) \oplus (X \otimes Z)$}
\DisplayProof
\AxiomC{$(Y \oplus Z) \otimes X$}
\RightLabel{Distrib}
\UnaryInfC{$(X \otimes Y) \oplus (X \otimes Z)$}
\end{prooftree}

The sum product network (SPN) is a commutative semiring over univariate distributions~\citep{friesen2016sum}:

\begin{center}
\begin{tabular}{ccc}
$PC \rightarrow v \sim \mathcal{D}$ &
$PC \rightarrow PC \oplus PC$ &
$PC \rightarrow PC \otimes PC$
\end{tabular}
\end{center}

Given a BN, we can compile it to a SPN using the following procedure from Butz (2019):

% Topsort: https://epubs.siam.org/doi/pdf/10.1137/0210049#page=17

\begin{algorithm}
\begin{algorithmic}[1]
\Procedure{BayesNetToSPN}{$bn$: BN}: SPN
\State $ac \leftarrow $\Call{VariableEliminate}{$bn$}
\State $ac \leftarrow $\Call{RedistributeParameters}{$ac$}
\State $ac \leftarrow $\Call{CompileMarginalized}{$ac$}
\State \Return{\Call{ReduceCircuit}{$ac$}}
\EndProcedure\\
\Procedure{ReduceCircuit}{$ac_0$: AC}: SPN
\State $ac_1 \leftarrow $ \Call{AddTerminalNodes}{$ac_0$}
\State $ac_1 \leftarrow $ \Call{MergeProducts}{$ac_1$}
\If{$ac_0 = ac_1$}
\State \Return{$ac_0$}
\Else
\State \Return{\Call{ReduceCircuit}{$ac_1$}}
\EndIf
\EndProcedure
\end{algorithmic}
\end{algorithm}

\newpage

\mysection{Latent variables are interpretable}

\vspace{0.5\baselineskip}

%\begin{center}
%\input{../misc/fig_interactome.tex}
%\end{center}

\vspace{\baselineskip}

Sorting the adjacency matrix of the protein interactome using the (approximate) MAP values of the $\AHvar_i$ reveals interpretable structure in the data.
The higher density of edges along the diagonal reveals homophily.
The block structure in the top left reveals stochastic equivalence.
These features can also be seen in the MAP $\Theta$.

\vspace{0.5\baselineskip}

\mysection{DSL}

\vspace{\baselineskip}

\begin{center}
\begin{tabular}{lccl}
%    \toprule
\multicolumn{4}{c}{Graph data}\\
\midrule
%\addlinespace[2pt]
%\textcolor{red}{Random function model}
Random function model & $\AHfunction$ & $\sim$ & $\GP\,(0, \kernel)$\\
Latent class & $\larray_{ij}$ & $=$ & $\Lambda_{\AHvar_i\AHvar_j}\,\textrm{where} \,\AHvar_i \in \{1,\ldots,K\}$\\
IRM & $\larray_{ij}$ & $=$ & $\Lambda_{\AHvar_i\AHvar_j}\,\textrm{where} \,\AHvar_i \in \{1,\ldots,\infty\}$\\
Latent distance & $\larray_{ij}$ & $=$ & $-|\AHvar_i - \AHvar_j|$\\
Eigenmodel &$\larray_{ij}$ & $=$ & $\AHvar_i'\Lambda \AHvar_j$\\
LFRM & $\larray_{ij}$ & $=$ & $\AHvar_i'\Lambda \AHvar_j\,\textrm{where} \,\AHvar_i \in \{0,1\}^\infty$\\
ILA & $\larray_{ij}$ & $=$ & $\sum_d \mathbb{I}_{U_{id}}\mathbb{I}_{U_{jd}}\Lambda^{(d)}_{U_{id}U_{jd}}\,\textrm{where} \,\AHvar_i \in \{0,\ldots,\infty\}^\infty$\\
SMGB & $\AHfunction$ & $\dist$ & $\GP\,(0, \kernel_1 \otimes \kernel_2)$ \\
%\midrule
\addlinespace[4pt]
\multicolumn{4}{c}{Real-valued array data}\\
\midrule
%\textcolor{red}{Random function model}
Random function model & $\AHfunction$ & $\sim$ & $\GP\,(0, \kernel)$\\
Mondrian process based & $\AHfunction$ & = & piece-wise constant random function\\
PMF & $\larray_{ij}$ & $=$ & $\AHvar_i'V_j$\\
GPLVM & $\AHfunction$ & $\sim$ & $\GP\,(0, \kernel \otimes \delta)$\\
%Linear relational GP~\cite{Yu2008} & $\AHfunction$ & $\sim$ & $\GP\,(0, \kernel_1 \otimes \kernel_2)\,\textrm{where} \,\kernel_i\,\textrm{is linear} $\\
%\bottomrule
\end{tabular}
\end{center}

\vspace{\baselineskip}

\mysection{Results}

\vspace{\baselineskip}

\begin{center}
\begin{tabular}{r | r r r | r r r | r r r}
\multicolumn{10}{c}{AUC results} \\
\addlinespace[2pt]
Data set & \multicolumn{3}{c|}{High school} & \multicolumn{3}{c|}{NIPS} & \multicolumn{3}{c}{Protein} \\
Latent dim. & 1 & 2 & 3 & 1 & 2 & 3 & 1 & 2 & 3 \\
\midrule
PMF                   & 0.747 & 0.792 & 0.792 & 0.729 & 0.789 & 0.820 & 0.787 & 0.810 & 0.841 \\
Eigenmodel            & 0.742 & 0.806 & 0.806 & 0.789 & 0.818 & 0.845 & 0.805 & 0.866 & 0.882 \\
GPLVM                 & 0.744 & 0.775 & 0.782 & 0.888 & 0.876 & 0.883 & 0.877 & 0.883 & 0.873 \\
RFM & \textbf{0.815} & \textbf{0.827} & \textbf{0.820} & \textbf{0.907} & \textbf{0.914} & \textbf{0.919} & \textbf{0.903} & \textbf{0.910} & \textbf{0.912}
\end{tabular}
\end{center}

\vspace{30pt}

\begin{tabular}{cc}
\begin{minipage}[c]{0.8\columnwidth}

Code available at: \url{https://github.com/breandan/markovian} \\

Paper available at: \url{https://brea.ndan.co/public/probcirc.pdf}
\end{minipage}
&
\begin{minipage}[c]{0.2\columnwidth}
\begin{centering}
\qrcode[height=2in]{kg.ndan.co}
\end{centering}
\end{minipage}
\end{tabular}

%\small{
%\bibliographystyle{unsrt}
%\bibliographystyle{../misc/natbib}
%\bibliography{../misc/library,../misc/biblio,../misc/bibdesk-porbanz}
%}

\end{multicols}

\end{poster}

\end{document}
