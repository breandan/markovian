%%%%%%%% ICML 2018 EXAMPLE LATEX SUBMISSION FILE %%%%%%%%%%%%%%%%%

\documentclass{article}

% Recommended, but optional, packages for figures and better typesetting:
\usepackage{microtype}
\usepackage{graphicx}
\usepackage{subfigure}
\usepackage{booktabs} % for professional tables
\usepackage{enumitem}

\usepackage{bussproofs}

\usepackage{sourcecodepro}
\usepackage{listings}
\usepackage{amsfonts}
\usepackage{tikz-qtree}
\usepackage{amsthm}
\usepackage{bm}
\usetikzlibrary{bayesnet}
\usetikzlibrary{arrows}
\usepackage{caption}
\usepackage{subcaption}
\usetikzlibrary{backgrounds}

\usepackage{mathtools}% superior to amsmath
\usepackage{tikz}
\makeatletter
\newcommand\ccirc[1]{%
\mathpalette\@ccirc{#1}%
}
\newcommand\@ccirc[2]{%
\tikz[baseline=(math.base)] \node[draw,circle, inner sep=1pt] (math) {$\m@th#1#2$};%
}
\newcommand\gcirc[1]{%
\mathpalette\@gcirc{#1}%
}
\newcommand\@gcirc[2]{%
\tikz[baseline=(math.base)] \node[draw,circle, fill=gray!30, inner sep=1pt] (math) {$\m@th#1#2$};%
}
\makeatother


% hyperref makes hyperlinks in the resulting PDF.
% If your build breaks (sometimes temporarily if a hyperlink spans a page)
% please comment out the following usepackage line and replace
% \usepackage{icml2018} with \usepackage[nohyperref]{icml2018} above.
\usepackage{hyperref}

% Attempt to make hyperref and algorithmic work together better:
\newcommand{\theHalgorithm}{\arabic{algorithm}}

% Use the following line for the initial blind version submitted for review:
%\usepackage{icml2018_ift6269}

% If accepted, instead use the following line for the camera-ready submission:
\usepackage[accepted]{icml2018_ift6269}
% SLJ: -> use this for your IFT 6269 project report!

% The \icmltitle you define below is probably too long as a header.
% Therefore, a short form for the running title is supplied here:
\icmltitlerunning{IFT 6269 Final Project: Probabilistic Reasoning, from Graphs to Circuits}

\begin{document}

\twocolumn[
\icmltitle{IFT 6269 Final Project: Probabilistic Reasoning, from Graphs to Circuits}

% It is OKAY to include author information, even for blind
% submissions: the style file will automatically remove it for you
% unless you've provided the [accepted] option to the icml2018
% package.

% List of affiliations: The first argument should be a (short)
% identifier you will use later to specify author affiliations
% Academic affiliations should list Department, University, City, Region, Country
% Industry affiliations should list Company, City, Region, Country

% You can specify symbols, otherwise they are numbered in order.
% Ideally, you should not use this facility. Affiliations will be numbered
% in order of appearance and this is the preferred way.
\icmlsetsymbol{equal}{*}

\begin{icmlauthorlist}
\icmlauthor{Breandan Considine}{socs,kast,mila}
\end{icmlauthorlist}

\icmlaffiliation{socs}{McGill University, School of Computer Science}
\icmlaffiliation{kast}{Knowledge and Software Technology Lab}
\icmlaffiliation{mila}{Mila Queb\'ec}

\icmlcorrespondingauthor{Breandan Considine}{breandan.considine@mail.mcgill.ca}

% You may provide any keywords that you
% find helpful for describing your paper; these are used to populate
% the "keywords" metadata in the PDF but will not be shown in the document
\icmlkeywords{Machine Learning, ICML}

\vskip 0.3in
]

% this must go after the closing bracket ] following \twocolumn[ ...

% This command actually creates the footnote in the first column
% listing the affiliations and the copyright notice.
% The command takes one argument, which is text to display at the start of the footnote.
% The \icmlEqualContribution command is standard text for equal contribution.
% Remove it (just {}) if you do not need this facility.

%\printAffiliationsAndNotice{}  % leave blank if no need to mention equal contribution
\printAffiliationsAndNotice{} % otherwise use the standard text.

\begin{abstract}
    Graphical models (PGMs) are very expressive, but even approximate inference on belief networks (BNs) is NP-hard~\citep{dagum1993approximating}. We can faithfully represent a large class of PGMs and their corresponding distributions as probabilistic circuits (PCs)~\citep{choi2020probabilistic}, which are capable of exact inference in polynomial time and tractable to calibrate using SGD or EM. PCs share many algebraic properties with PGMs and can propagate statistical estimators using simple algebraic rules. In this work, we will see how to compile BNs to PCs using the approach developed by \citet{zhao2015relationship}, and demonstrate their equivalence on a few toy inference problems.

\end{abstract}

\section{Introduction}\label{sec:intro}

Graphical models~\citep{jordan2003introduction,koller2009probabilistic} are a framework for modeling probability distributions over variables whose conditional dependence structure can be expressed as a graph. In general, inference on PGMs is NP-hard~\citep{cooper1990computational}. Recent work~\citep{choi2020probabilistic} has explored computationally tractable models for probabilistic reasoning based on semiring algebras. Semirings are known to have many useful applications in graph theory~\citep{dolan2013fun} and formal languages~\citep{bernady2013efficient}.

Let $\Omega$ be a set of events and $\Sigma$ a collection of subsets of $\Omega$. A probability distribution is a function $P: \Sigma \rightarrow \mathbb{R}^{+}$ which satisfies the~\citet{kolmogorov1933grundbegriffe} axioms, in particular:

\begin{itemize}
    \itemsep-1em
    \item [(4)] $P(\Omega) = 1$ & \\
    \item [(5)] $P(A + B) = P(A) + P(B)$, when $A \cap B = \emptyset$
\end{itemize}

% https://math.stackexchange.com/questions/3573008/kolmogorov-probability-theory-question
% https://archive.org/details/foundationsofthe00kolm/page/2/mode/2up

The simplest distribution assigns equal probability to all $\omega \in \Omega$. This is called the uniform distribution, $\mathcal{U}$.

$$
\mathcal U(\omega_1) = \mathcal U(\omega_2)\text{, } \forall \omega_1, \omega_2 \in \Omega
$$

Some common exponential family distributions include:

\begin{tabular}{cccc}
    $\mathcal{D} \rightarrow \mathtt{Normal}$ &
    $\mathcal{D} \rightarrow \mathtt{Bernoulli}$ &
    $\mathcal{D} \rightarrow \mathtt{Dirichlet}$ &
\end{tabular}

We can join two distributions to form a \textit{joint distribution}:

\begin{center}
\begin{tabular}{ccc}
    $\mathcal{D} \rightarrow \mathcal{D}, \mathcal{D}$ &$\mathcal{J} \rightarrow P(\mathcal{D})$ &$\mathcal{J} \rightarrow \mathcal{J}\mathcal{J}$
\end{tabular}
\end{center}


%We can assign a probability distribution to a variable, by \textit{sampling} from it. As we increase the sample size, the distribution of values will converge to the true distribution.
%
%$$
%d \sim \mathcal{D}
%$$

The joint distribution, $P(X, Y)$ is a new distribution over the Cartesian product of the two input spaces, $X \times Y$:

\begin{prooftree}
    \AxiomC{$P(X): X\rightarrow \mathbb R^{+}$}
    \AxiomC{$P(Y): Y\rightarrow \mathbb R^{+}$}
    \RightLabel{JOIN}
    \BinaryInfC{$P(X, Y): X \times Y \rightarrow \mathbb R^{+}$}
\end{prooftree}

If we have a joint distribution $P(X, Y)$, and observe a specific event $y: Y$, this operation is called \textit{conditioning} and the resulting distribution over $X$ is called a \textit{conditional}:

% https://www.cambridge.org/core/services/aop-cambridge-core/content/view/819623B1B5B33836476618AC0621F0EE/9781108488518AR.pdf/Foundations_of_Probabilistic_Programming.pdf?event-type=FTLA#page=325

\begin{prooftree}
    \AxiomC{$P(X, Y): X \times Y\rightarrow\mathbb{R}^+$}
    \AxiomC{$y: Y$}
    \RightLabel{COND}
    \BinaryInfC{$P(X \mid Y = y): X\rightarrow\mathbb{R}^+$}
\end{prooftree}

We may decompose a joint distribution $P(X, Y)$ by \textit{marginalizing}, or summing over all possible conditionals. The resulting distribution is called a \textit{marginal}:

\begin{prooftree}
    \AxiomC{$P(X, Y): X\times Y\rightarrow\mathbb{R}^+$}
    \RightLabel{MARG}
    \UnaryInfC{$P(X) \propto \sum_{y \in Y} P(X \mid Y = y)$}
\end{prooftree}

When a conditional distribution $P(X \mid Y)$ does not depend on the conditioned event, $X$ and $Y$ are called \textit{independent}.

\begin{prooftree}
    \AxiomC{$P(X \mid Y) = P(X)$}
    \RightLabel{IND}
    \UnaryInfC{$X \perp Y$}
\end{prooftree}


Equivalently, when two distributions $P(X)$ and $P(Y)$ are multiplied to form a joint distribution $P(X, Y)$, we may also conclude $X$ and $Y$ are independent:

\begin{prooftree}
    \AxiomC{$P(X, Y) = P(X)P(Y)$}
    \RightLabel{IND}
    \UnaryInfC{$X \perp Y$}
\end{prooftree}

When two conditionals $P(X \mid Z)$, $P(Y \mid Z)$ are multiplied to form a joint distribution $P(X, Y \mid Z)$, $X$ and $Y$ are said to be \textit{conditionally independent given $Z$}:

\begin{prooftree}
    \AxiomC{$P(X,Y\mid Z) = P(X \mid Z)P(Y \mid Z)$}
    \RightLabel{CIND}
    \UnaryInfC{$X\perp Y \mid Z$ }
\end{prooftree}

%To draw a single sample from a distribution $\mathcal{D}$, we can do so as follows:
%
%$$
%
%$$

Following \citet{pearl1985graphoids}, the grammar of conditional independence statements has some equivalence relations:

% http://ftp.cs.ucla.edu/pub/stat_ser/r53-L.pdf#page=8

\begin{prooftree}
    \AxiomC{$X \perp Y \mid Z$}
    \RightLabel{SYM}
    \UnaryInfC{$Y \perp X \mid Z$}
    \DisplayProof
    \hskip 1.5em
    \AxiomC{$X \perp Y, W \mid Z$}
    \RightLabel{DCOMP}
    \UnaryInfC{$X \perp Y \mid Z$}
    \DisplayProof

    \AxiomC{$X \perp Y \mid Z$}
    \AxiomC{$X \perp Z \mid Y$}
    \RightLabel{UNION}
    \BinaryInfC{$X \perp Y,W \mid Z$}
    \DisplayProof

    \AxiomC{$X \perp W \mid Y, Z$}
    \AxiomC{$X \perp Z \mid Y$}
    \RightLabel{CONTRACT}
    \BinaryInfC{$X \perp W \mid Y$}
\end{prooftree}

A directed graphical model (DGM) is a graph whose edges capture conditional independence relations between RVs.

% http://maximustann.github.io/mach/2015/07/06/belief-network-2/
% http://frnsys.com/notes/ai/foundations/probabilistic_graphical_models.html

\begin{prooftree}
    \AxiomC{$X \not\perp Y \mid Z$}
    \RightLabel{CLDR}
    \UnaryInfC{$\ccirc{X}\rightarrow\gcirc{Z}\leftarrow\ccirc{Y}$}
    \DisplayProof
    \hskip .6em
    \AxiomC{$X \perp Y \mid Z$}
    \RightLabel{FORK}
    \UnaryInfC{$\ccirc{X}\leftarrow\gcirc{Z}\rightarrow\ccirc{Y}$}
    \DisplayProof

    \AxiomC{$X \perp Y \mid Z$}
    \RightLabel{CHAIN}
    \UnaryInfC{$\ccirc{X}\rightarrow\gcirc{Z}\rightarrow\ccirc{Y}$}
\end{prooftree}

A belief network (BN) is an acyclic DGM of the form:

\begin{equation}
    P(x_1,\ldots,x_D)=\prod_{i=1}^D P(x_i \mid \texttt{parents}(x_i))
\end{equation}

A path between two vertices $\ccirc{A} \ldots \ccirc{B}$ is blocked if:

\begin{enumerate}[(a)]
    \item $\ccirc{A} \ldots \rightarrow \gcirc{V} \rightarrow \ldots \ccirc{B}$ or $\ccirc{A}\ldots\leftarrow\gcirc{V}\rightarrow\ldots\ccirc{B}$.
    \item $\ccirc{A}\ldots\rightarrow \ccirc{V} \leftarrow\ldots\ccirc{B}$ and $\gcirc{?} \not\in \texttt{desc}(\ccirc{V})$.
\end{enumerate}

$\ccirc{A}$ and $\ccirc{B}$ are d-separated if all paths are blocked.

A probabilistic circuit (PC) is a semiring over distributions:

\begin{center}
    \begin{tabular}{ccc}
        $PC \rightarrow v \sim \mathcal{D}$ &
        $PC \rightarrow PC + PC$ &
        $PC \rightarrow PC \times PC$
    \end{tabular}
\end{center}

To obtain statistical estimators for $PC$:

TODO: Moralization as "marrying" the parents

TODO: Plate notation

% https://en.wikipedia.org/wiki/Propagation_of_uncertainty

We can compile a tree BN to a PC as follows: \ldots

%\begin{center}
%    \begin{tabular}{cc}
%        \begin{figure}
%            \tikz{
%            \node[obs] (z) {$z$};%
%            \node[latent,above=of z,xshift=-1cm,fill] (x) {$x$}; %
%            \node[latent,above=of z,xshift=1cm] (y) {$y$}; %
%            \edge {x,y} {z}  }
%        \end{figure} & \begin{figure}
%                           \tikz{
%                           \node[obs] (z) {$z$};%
%                           \node[latent,above=of z,xshift=-1cm,fill] (x) {$x$}; %
%                           \node[latent,above=of z,xshift=1cm] (y) {$y$}; %
%                           \edge {z} {x,y}   } \end{figure}
%    \end{tabular} \\
%    $P(X,Y|Z) \propto P(Z|X,Y)P(X)P(Y)$ & $P(X,Y|Z)=P(X|Z)P(Y|Z)$
%\end{center}

\bibliography{example_paper}
\bibliographystyle{icml2018}

\end{document}